\documentclass[uplatex]{jsarticle}

\usepackage{amssymb}
\usepackage{amsmath}
\usepackage{mathrsfs}
\usepackage{amsfonts}
\usepackage{mathtools}

\usepackage{xcolor}
\usepackage[dvipdfmx]{graphicx}



\usepackage{ulem}

\usepackage{braket}

%%%%%ハイパーリンク
%\usepackage[colorlinks=true,urlcolor=blue!70!black,citecolor=blue!60!black,linkcolor=blue!60!black]{hyperref}
%\usepackage{aliascnt} %for creating different biblatex references for different theoremstyles
\usepackage[setpagesize=false,dvipdfmx]{hyperref}
\usepackage{aliascnt}
\hypersetup{
    colorlinks=true,
    citecolor=blue,
    linkcolor=blue,
    urlcolor=blue,
}

\renewcommand{\eqref}[1]{\textcolor{blue}{(\ref{#1})}}

%%%%%%ハイパーリンク


%%%%%図式
%\usepackage{tikz}%%%図
\usepackage{amscd}%%%簡単な図式

\usepackage{tikz}
\usepackage{tikz-cd} %commutative diagrams in TikZ
\usetikzlibrary{calc}
\usetikzlibrary{matrix,arrows}
\usetikzlibrary{decorations.markings}

%%%%%図式



%%%%%%%%%%%%定理環境%%%%%%%%%%%%
%%%%%%%%%%%%定理環境%%%%%%%%%%%%
%%%%%%%%%%%%定理環境%%%%%%%%%%%%

\usepackage{amsthm}

\newcommand{\myTheoremEnvironments}[1]{#1}

\myTheoremEnvironments{
%%%%%%%%%%%%Plain型%%%%%%%%%%%%


%%%%%%%%%%%%definition型%%%%%%%%%%%%

\theoremstyle{definition}

\renewcommand{\sectionautorefname}{Section}

\newtheorem{thm}{Theorem}%[section]
\newcommand{\thmautorefname}{Theorem}


\newaliascnt{prop}{thm}%%%カウンター「prop」の定義(thmと同じ)
\newtheorem{prop}[prop]{Proposition}
\aliascntresetthe{prop}
\newcommand{\propautorefname}{Proposition}%%%カウンター名propは「命題」で参照する

\newaliascnt{cor}{thm}
\newtheorem{cor}[cor]{Corollary}
\aliascntresetthe{cor}
\newcommand{\corautorefname}{Corollary}

\newaliascnt{lem}{thm}
\newtheorem{lem}[lem]{Lemma}
\aliascntresetthe{lem}
\newcommand{\lemautorefname}{Lemma}

%%%%%%%アルファベットで番号づける定理環境
\newtheorem{thmA}{Theorem}[section]
\newcommand{\thmAautorefname}{Theorem}
\renewcommand\thethmA{\Alph{thmA}}

\newtheorem{corA}{Theorem}[section]
\newcommand{\corAautorefname}{Corollary}
\renewcommand\thecorA{\Alph{corA}}

\newaliascnt{defi}{thm}
\newtheorem{defi}[defi]{Definition}
\aliascntresetthe{defi}
\newcommand{\defiautorefname}{Definition}

\newaliascnt{rem}{thm}
\newtheorem{rem}[rem]{Remark}
\aliascntresetthe{rem}
\newcommand{\remautorefname}{Remark}

\newaliascnt{reconstruction}{thm}
\newtheorem{reconstruction}[reconstruction]{Reconstruction}
\aliascntresetthe{reconstruction}
\newcommand{\reconstructionautorefname}{Reconstruction}

%%%%%%%番号づけない定理環境
\newtheorem*{exam*}{Example}
\newtheorem*{rrem*}{Remark}
\newtheorem*{defi*}{Definition}
\newtheorem*{setting*}{Setting}
\newtheorem*{notation*}{Notations}


%%%%%%%%%%%%定理環境%%%%%%%%%%%%
%%%%%%%%%%%%定理環境%%%%%%%%%%%%
%%%%%%%%%%%%定理環境%%%%%%%%%%%%
}



%%%%%箇条書き環境
\usepackage[]{enumitem}

\makeatletter
\AddEnumerateCounter{\fnsymbol}{\c@fnsymbol}{9}%%%%fnsymbolという文字をenumerate環境のパラメーターで使えるようにする。
\makeatother

\makeatletter
\renewcommand{\p@enumii}{}
\makeatother

\renewcommand{\theenumi}{(\roman{enumi})}%%%%%itemは(1),(2),(3)で番号付ける。
\renewcommand{\labelenumi}{\theenumi}

\renewcommand{\theenumii}{(\alph{enumii})}%%%%%itemは(1),(2),(3)で番号付ける。
\renewcommand{\labelenumii}{\theenumii}

\usepackage{moreenum}


\makeatletter
\newcommand*{\@yuyuspadecount}[1]{\ensuremath{
\ifcase #1\or\spadesuit\or\spadesuit_2\or\spadesuit_3
\or\spadesuit_4\or\spadesuit_5\or\spadesuit_6
\or\spadesuit_7\or\spadesuit_8\or\spadesuit_9
\else\@ctrerr\fi\relax}}
\newcommand*{\yuyuspadecount}[1]{%
\expandafter\@yuyuspadecount\csname c@#1\endcsname
}
\AddEnumerateCounter{\yuyuspadecount}{\@yuyuspadecount}{9}

\newcommand*{\@yuyuclubcount}[1]{\ensuremath{
\ifcase #1\or\clubsuit_1\or\clubsuit_2\or\clubsuit_3
\or\clubsuit_4\or\clubsuit_5\or\clubsuit_6
\or\clubsuit_7\or\clubsuit_8\or\clubsuit_9
\else\@ctrerr\fi\relax}}
\newcommand*{\yuyuclubcount}[1]{%
\expandafter\@yuyuclubcount\csname c@#1\endcsname
}
\AddEnumerateCounter{\yuyuclubcount}{\@yuyuclubcount}{9}

\newcommand*{\@yuyustarcount}[1]{\ensuremath{
\ifcase #1\or\star_1\or\star_2\or\star_3
\or\star_4\or\star_5\or\star_6
\or\star_7\or\star_8\or\star_9
\else \@ctrerr \fi\relax}}
\newcommand*{\yuyustarcount}[1]{%
\expandafter\@yuyustarcount\csname c@#1\endcsname
}
\AddEnumerateCounter{\yuyustarcount}{\@yuyustarcount}{9}
\makeatother
%%%%%箇条書き環境



\usepackage{mandorasymb}
\usepackage{applekeys}
\renewcommand{\qedsymbol}{\pencilkey}
%\renewcommand{\qedsymbol}{\kinoposymbniko}




\usepackage{latexsym}

\newcommand{\myMacros}[1]{#1}

\myMacros{
\DeclareMathOperator{\Hom}{Hom}
\DeclareMathOperator{\Isom}{Isom}
\DeclareMathOperator{\ISOM}{\mathbf{Isom}}
\DeclareMathOperator{\id}{\mathrm{id}}
\DeclareMathOperator{\im}{\mathrm{Im}}

\DeclareMathOperator{\coker}{\mathrm{coker}}
\DeclareMathOperator{\colim}{\mathrm{colim}}
\DeclareMathOperator{\plim}{\mathrm{lim}}
\DeclareMathOperator{\rank}{\mathrm{rank}}
\DeclareMathOperator{\codim}{\mathrm{codim}}

\DeclareMathOperator{\Spec}{\mathrm{Spec}}
\DeclareMathOperator{\Proj}{\mathrm{Proj}}
\DeclareMathOperator{\Sym}{\mathrm{Sym}}
\DeclareMathOperator{\Ext}{\mathrm{Ext}}
\DeclareMathOperator{\Bs}{\mathrm{Bs}}
\DeclareMathOperator{\Bl}{\mathrm{Bl}}
\DeclareMathOperator{\Sing}{\mathrm{Sing}}
\DeclareMathOperator{\red}{\mathrm{red}}
\DeclareMathOperator{\Reg}{\mathrm{Reg}}
\DeclareMathOperator{\Ridge}{\mathrm{Ridge}}
\DeclareMathOperator{\Hilb}{\mathrm{Hilb}}
\DeclareMathOperator{\Grass}{\mathrm{Grass}}


\newcommand{\A}{\mathbb{A}}
\newcommand{\C}{\mathbb{C}}
\renewcommand{\P}{\mathbb{P}}
\newcommand{\R}{\mathbb{R}}
\newcommand{\Q}{\mathbb{Q}}
\newcommand{\Z}{\mathbb{Z}}
\newcommand{\N}{\mathbb{N}}



\newcommand{\mcA}{\mathcal{A}}
\newcommand{\mcB}{\mathcal{B}}
\newcommand{\mcC}{\mathcal{C}}
\newcommand{\mcD}{\mathcal{D}}
\newcommand{\mcE}{\mathcal{E}}
\newcommand{\mcF}{\mathcal{F}}
\newcommand{\mcG}{\mathcal{G}}
\newcommand{\mcH}{\mathcal{H}}
\newcommand{\mcI}{\mathcal{I}}
\newcommand{\mcJ}{\mathcal{J}}
\newcommand{\mcK}{\mathcal{K}}
\newcommand{\mcL}{\mathcal{L}}
\newcommand{\mcM}{\mathcal{M}}
\newcommand{\mcN}{\mathcal{N}}
\newcommand{\mcO}{\mathcal{O}}
\newcommand{\mcP}{\mathcal{P}}
\newcommand{\mcQ}{\mathcal{Q}}
\newcommand{\mcR}{\mathcal{R}}
\newcommand{\mcS}{\mathcal{S}}
\newcommand{\mcT}{\mathcal{T}}
\newcommand{\mcU}{\mathcal{U}}
\newcommand{\mcV}{\mathcal{V}}
\newcommand{\mcW}{\mathcal{W}}
\newcommand{\mcX}{\mathcal{X}}
\newcommand{\mcY}{\mathcal{Y}}
\newcommand{\mcZ}{\mathcal{Z}}

\DeclareMathOperator{\OOO}{\mcO}

\newcommand{\OB}{\OOO_B}
\newcommand{\OC}{\OOO_C}
\newcommand{\OD}{\OOO_D}
\renewcommand{\OE}{\OOO_E}
\newcommand{\OF}{\OOO_F}
\newcommand{\OH}{\OOO_H}
\newcommand{\OP}{\OOO_P}
\newcommand{\OQ}{\OOO_Q}
\newcommand{\OR}{\OOO_R}
\newcommand{\OS}{\OOO_S}
\newcommand{\OT}{\OOO_T}
\newcommand{\OU}{\OOO_U}
\newcommand{\OV}{\OOO_V}
\newcommand{\OW}{\OOO_W}
\newcommand{\OX}{\OOO_X}
\newcommand{\OY}{\OOO_Y}
\newcommand{\OZ}{\OOO_Z}

\newcommand{\OO}[1]{\OOO_{#1}}

\newcommand{\loc}{\mathrm{loc}}

\newcommand{\rsa}{\rightsquigarrow}
\newcommand{\dto}{\dashrightarrow}
\renewcommand{\emptyset}{\varnothing}

\newcommand{\dfn}{:\overset{\mathrm{\scriptsize def}}{=}}
}


\newcommand{\lt}{<}
\renewcommand{\gt}{>}



\newcommand{\HTMLSkip}[1]{#1}


\newcommand{\HereBeginTikz}{}
\newcommand{\HereEndTikz}{}
\newcommand{\HTMLhead}[1]{}

\HTMLhead{
---
layout: article-type
title: "Strange Curves"
category: Notes
tag: "Algebraic Geometry"
author: Yujitomo
description: "Strangeな曲線に関するノートです"
---
}


%%%%%%%%%タイトル
\title{Strange Curves}
\author{ゆじ}
\date{\today}

\begin{document}
\maketitle


このノートでは、Strangeな非特異射影曲線の構造に関するSamuelの定理
\cite[定理 IV.3.9]{Ha}に、Hartshorneによるものとは別の証明を与える。

\begin{defi}
  \(k\)を代数閉体、\(C\subset \P^n\)を射影的な曲線とする。
  \(C\)のすべての正則な点での接線が同じ点\(p\in \P^n\)を通るとき、
  \(C\)を (埋め込みのもと、\(\P^n\)内で、) \textbf{Strange}であるという。
\end{defi}


\begin{notation*}
  \HTMLSkip{ \ }
  \begin{itemize}
    \item
    スキームの射\(f:T\to S\)と\(S\)上の対象\(F\)
    (\(S\)-スキームや、\(S\)上のスキームの射や、\(S\)上の準連接層など)
    に対し、\(F_T\)で\(F\)の射\(T\to S\)による基底変換を表す。
    \item
    \(k\)を代数閉体とする。
    \item
    \(k\)-線形空間\(V\)に対し、
    \(\P(V) \dfn \Proj(\Sym(V))\)と書く。
    \(\OO{\P(V)}(1)\)を\(\P(V)\)上のトートロジカル直線束とする。
    \item
    \(\P^n\)と\(\P(H^0(\P^n,\OO{\P^n}(1)))\)は自然に同型なので、
    このノートではこれらを同一視する。
    \item
    \(k\)上の代数多様体\(X\)に対し、
    \(\Delta_{(1)}\)で対角射\(X\to X\times_kX\)の一次無限小近傍、
    つまりイデアル層\(\mcI_{\Delta}^2\)に対応する\(X\times_kX\)の閉部分スキームとする。
    第一、第二射影を
    \(\mathrm{pr}_1,\mathrm{pr}_2:X\times_kX\rightrightarrows X\)と書き、
    \(p_1,p_2:\Delta_{(1)}\rightrightarrows X\)をそれぞれ
    \(\mathrm{pr}_1,\mathrm{pr}_2\)と
    閉埋め込み\(\Delta_{(1)}\to X\times_kX\)の合成とする。
    代数多様体\(X\)上の準連接層\(\mcF\)に対し、
    \(\mcP^1(\mcF)\dfn p_{2,*}p_1^*\mcF\)と置く。
  \end{itemize}
\end{notation*}

\begin{rem}
  \HTMLSkip{ \ }
  \begin{itemize}
    \item
    代数多様体\(X\)上の準連接層\(\mcF\)に対し、
    平坦基底変換により
    \(\mathrm{pr}_{2,*}\mathrm{pr}_1^*\mcF\cong H^0(X,\mcF)\otimes_k \OX\)であるから、
    射の列\(H^0(X,\mcF)\otimes_k \OX\to p_{2,*}p_1^*\mcF\to \mcF\)ができる。
    \item
    \(V\)を有限次元\(k\)-線形空間とする。
    代数多様体\(X\)から\(\P(V)\)への射は、
    \(X\)上の直線束\(L\)への全射\(V_X\to L\)と対応する
    (cf. \cite[Theorem II.7.12]{Ha})。
    射\(V_X\to L\)は\(k\)-線形空間の射\(V\to H^0(X,L)\)と対応し、
    これにより\(V_X\to \mcP^1(L)\)を引き起こす。
    \(X\to \P(V)\)が閉埋め込みであれば
    射\(V_X\to \mcP^1(L)\)は全射となる
    (cf. \cite{YJ})。
    \item
    \(V\)を有限次元\(k\)-線形空間とする。
    \(X\subset \P(V)\)を射影代数多様体とすると、
    \(X\)上で直線束\(L=\OO{\P(V)}(1)|_X\)と全射\(V_X\to \mcP^1(L)\)を得る。
    閉点\(x\in X\)を正則点とする。
    全射\(V_X\to \mcP^1(L)\)を点\(x\)へ基底変換すると、
    全射\(V\to k(x)\oplus \mathfrak{m}_x/\mathfrak{m}_x^2\)を得る。
    この全射が定める線形部分多様体
    \(\P(k(x)\oplus \mathfrak{m}_x/\mathfrak{m}_x^2)\subset \P(V)\)は、
    \(X\)の点\(x\in X\)での接平面 (embedded tangent plane) である。
  \end{itemize}
\end{rem}




\begin{thm}[{\cite[定理 IV.3.9]{Ha}}]
  \(k\)を代数閉体、\(C\subset \P^n\)を非特異射影曲線とする。
  このとき\(C\cong \P^1\)であり、
  さらに\(C\)は\(\P^n\)内の直線か、または、
  ある平面\(\P^2\subset \P^n\)に含まれる次数\(2\)の曲線のいずれかとなる。
\end{thm}


\begin{proof}
  \(g\)を\(C\)の種数、\(d\)を\(C\)の次数とする。
  \(g = 0,d\leq 2\)を示せば良い。
  \(V\dfn H^0(\P^n,\OO{\P^n}(1))\)と置き、\(\P^n=\P(V)\)と書く。
  \(L = \OO{\P(V)}(1)|_C\)と置く。
  全射の列\(V_C\to \mcP^1(L) \to L\)ができ、
  埋め込み\(C\subset \P(V)\)は全射\(V_C\to L\)により引き起こされている。

  \(C\)のすべての接線が通る点を\(p\)とし、
  点\(p\in \P(V)\)を与える全射も同じ記号\(p:V\to k\)で表す。
  \(C\)のすべての接線が点\(p\)を通ることは、
  \[
  \ker(V_C\to \mcP^1(L)) \subset \ker(V_C\xrightarrow{p_C} k_C)
  \]
  を意味し、
  従って全射\(p_C : V_C\to k_C\)は\(V_C\to \mcP^1(L)\)を経由して分解する。
  こうして全射\(\mcP^1(L)\to k_C\)を得る。
  一方で、自然な全射\(\mcP^1(L)\to L\)もあるが、
  \(L\not\cong k_C\)であることから、
  二つの射\(\mcP^1(L)\to k_C\)と\(\mcP^1(L)\to L\)の核はたがいに他を含まない。
  従って、これらの射を並べて得られる射\(\mcP^1(L) \to L\oplus k_C\)は単射となる。
  \(\det\)を取れば直線束の単射\(\det(\mcP^1(L)) \to \det(L\oplus k_C) \cong L\)を得る。
  完全列
  \[
  \begin{CD}
    0 @>>> \Omega_X\otimes L @>>> \mcP^1(L) @>>> L @>>> 0
  \end{CD}
  \]
  より\(\det(\mcP^1(L)) \cong \Omega_X \otimes L^{\otimes 2}\)であり
  \(\deg(\det(\mcP^1(L))) = 2g - 2 + 2d\)となる。
  従って不等式
  \[
  \deg(\det(\mcP^1(L))) \ = \ 2g - 2 + 2d \ \leq \ \deg(L) \ = \ d
  \]
  を得る。
  これを実現する整数\(g\geq 0, d\geq 1\)の組は
  \[
  (g,d) \ = \ (0,1) \ , \ (0,2)
  \]
  しかありえない。
\end{proof}


\begin{rem}
  ある平面\(\P^2\subset \P^n\)に含まれる次数\(2\)の
  非特異射影曲線\(C\)がstrangeであるとする。
  点\(p\in \P^2\)を、\(C\)のすべての接線が通る点とする。
  \(C\)は次数\(2\)であるから、\(p\)を通る直線は\(C\)と必ず接する。
  よって、点\(p\)から\(\P^1\)へ射影すると、
  次数\(2\)の単射\(f:C\to \P^1\)を得る。
  このとき\(f\)は純非分離であり、次数が\(2\)であることから、
  標数\(2\)でなければならないことがわかる。
  特に、標数\(p\neq 2\)のstrangeな非特異射影曲線\(C\subset \P^n\)は直線しかあり得ない。
\end{rem}



曲線が特異点を持つ場合には、標数正であれば、strangeな曲線はたくさんあり得る。
例は\cite[演習 IV.3.8.(a)]{Ha}に載っている通りである。
一方、その次の演習問題\cite[演習 IV.3.8.(b)]{Ha}にある通り、
標数\(0\)ではstrangeな曲線は直線しかあり得ない。


\begin{thm}[{\cite[演習 IV.3.8.(b)]{Ha}}]
  \(k\)を標数\(0\)の代数閉体、
  \(C\subset \P^n\)を (非特異とは限らない) 射影的でstrangeな曲線とする。
  このとき\(C\)は直線である。
\end{thm}

\begin{proof}
  strangeな (非特異とは限らない射影的な) 曲線\(C\subset \P^n\)は、
  点からの射影を繰り返すことにより、
  低い次元の射影空間内のstrangeな曲線と双有理である
  (cf. \cite[演習 I.4.9.]{Ha})。
  よって、射\(f:C\to \P^2\)であって以下を満たすものが存在する:
  \begin{itemize}
    \item
    \(f\)は像への双有理射である。
    \item
    \(f\)の像は\(\P^2\)内でstrangeである。
  \end{itemize}
  よって、\(\P^2\)内のstrangeな (非特異とは限らない射影的な) 曲線\(C\subset \P^2\)が
  直線に限ることを示せば良い。

  \(\im(f)\)のすべての正則点での接線が通る点を\(p\in \P^2\)と置き、
  \(C\)の正規化を\(\sigma:\tilde{C}\to C\)と置く。
  点\(p\)からの射影\(\P^2\dto \P^1\)と\(f\circ \sigma\)を合成することで、
  射\(g:\tilde{C}\to \P^1\)を得る。
  もし\(g\)が一点に潰れるならば、
  \(C\)はその点のfiber、つまりある\(\P^2\)内の直線に含まれるので、
  \(C\)は直線となることがわかる。
  そうでない場合、
  \(g\)は\(C\)の正則点に対応する
  \(\tilde{C}\)の点 (これは無限個ある) の上で分岐する。
  標数\(0\)であるので分岐点は有限個でなければならず、これは矛盾である。
  以上で示された。
\end{proof}






\begin{thebibliography}{9}
  \bibitem[Ha]{Ha}
  R.Hartshorne,
  \textit{Algebraic Geometry}.
  Springer-Verlag, New Tork, 1977. Graduate Text in Mathematics, No. 52.
  \bibitem[ゆ]{YJ}
  ゆじノート,
  \textit{Separating Tangent Vectors}.
\end{thebibliography}
\end{document}
