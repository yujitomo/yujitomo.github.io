\documentclass[uplatex]{jsarticle}

\usepackage{amssymb}
\usepackage{amsmath}
\usepackage{mathrsfs}
\usepackage{amsfonts}
\usepackage{mathtools}

\usepackage{xcolor}
\usepackage[dvipdfmx]{graphicx}



\usepackage{ulem}

\usepackage{braket}

%%%%%ハイパーリンク
%\usepackage[colorlinks=true,urlcolor=blue!70!black,citecolor=blue!60!black,linkcolor=blue!60!black]{hyperref}
%\usepackage{aliascnt} %for creating different biblatex references for different theoremstyles
\usepackage[setpagesize=false,dvipdfmx]{hyperref}
\usepackage{aliascnt}
\hypersetup{
    colorlinks=true,
    citecolor=blue,
    linkcolor=blue,
    urlcolor=blue,
}

\renewcommand{\eqref}[1]{\textcolor{blue}{(\ref{#1})}}

%%%%%%ハイパーリンク


%%%%%図式
%\usepackage{tikz}%%%図
\usepackage{amscd}%%%簡単な図式

\usepackage{tikz}
\usepackage{tikz-cd} %commutative diagrams in TikZ
\usetikzlibrary{calc}
\usetikzlibrary{matrix,arrows}
\usetikzlibrary{decorations.markings}

%%%%%図式



%%%%%%%%%%%%定理環境%%%%%%%%%%%%
%%%%%%%%%%%%定理環境%%%%%%%%%%%%
%%%%%%%%%%%%定理環境%%%%%%%%%%%%

\usepackage{amsthm}

%%%%%%%%%%%%Plain型%%%%%%%%%%%%


%%%%%%%%%%%%definition型%%%%%%%%%%%%

\theoremstyle{definition}

\renewcommand{\sectionautorefname}{Section}

\newtheorem{thm}{Theorem}[section]
\newcommand{\thmautorefname}{Theorem}


\newaliascnt{prop}{thm}%%%カウンター「prop」の定義(thmと同じ)
\newtheorem{prop}[prop]{Proposition}
\aliascntresetthe{prop}
\newcommand{\propautorefname}{Proposition}%%%カウンター名propは「命題」で参照する

\newaliascnt{cor}{thm}
\newtheorem{cor}[cor]{Corollary}
\aliascntresetthe{cor}
\newcommand{\corautorefname}{Corollary}

\newaliascnt{lem}{thm}
\newtheorem{lem}[lem]{Lemma}
\aliascntresetthe{lem}
\newcommand{\lemautorefname}{Lemma}

%%%%%%%アルファベットで番号づける定理環境
\newtheorem{thmA}{Theorem}[section]
\newcommand{\thmAautorefname}{Theorem}
\renewcommand\thethmA{\Alph{thmA}}

\newtheorem{corA}{Theorem}[section]
\newcommand{\corAautorefname}{Corollary}
\renewcommand\thecorA{\Alph{corA}}

\newaliascnt{defi}{thm}
\newtheorem{defi}[defi]{Definition}
\aliascntresetthe{defi}
\newcommand{\defiautorefname}{Definition}

\newaliascnt{rem}{thm}
\newtheorem{rem}[rem]{Remark}
\aliascntresetthe{rem}
\newcommand{\remautorefname}{Remark}

\newaliascnt{reconstruction}{thm}
\newtheorem{reconstruction}[reconstruction]{Reconstruction}
\aliascntresetthe{reconstruction}
\newcommand{\reconstructionautorefname}{Reconstruction}

%%%%%%%番号づけない定理環境
\newtheorem*{exam*}{Example}
\newtheorem*{rrem*}{Remark}
\newtheorem*{defi*}{Definition}
\newtheorem*{setting*}{Setting}
\newtheorem*{notation*}{Notations}
\newtheorem*{Ex*}{Exercise}


%%%%%%%%%%%%定理環境%%%%%%%%%%%%
%%%%%%%%%%%%定理環境%%%%%%%%%%%%
%%%%%%%%%%%%定理環境%%%%%%%%%%%%





%%%%%箇条書き環境
\usepackage[]{enumitem}

\makeatletter
\AddEnumerateCounter{\fnsymbol}{\c@fnsymbol}{9}%%%%fnsymbolという文字をenumerate環境のパラメーターで使えるようにする。
\makeatother

\makeatletter
\renewcommand{\p@enumii}{}
\makeatother

\renewcommand{\theenumi}{(\roman{enumi})}%%%%%itemは(1),(2),(3)で番号付ける。
\renewcommand{\labelenumi}{\theenumi}

\renewcommand{\theenumii}{(\alph{enumii})}%%%%%itemは(1),(2),(3)で番号付ける。
\renewcommand{\labelenumii}{\theenumii}

\usepackage{moreenum}


\makeatletter
\newcommand*{\@yuyuspadecount}[1]{\ensuremath{
\ifcase #1\or\spadesuit\or\spadesuit_2\or\spadesuit_3
\or\spadesuit_4\or\spadesuit_5\or\spadesuit_6
\or\spadesuit_7\or\spadesuit_8\or\spadesuit_9
\else\@ctrerr\fi\relax}}
\newcommand*{\yuyuspadecount}[1]{%
\expandafter\@yuyuspadecount\csname c@#1\endcsname
}
\AddEnumerateCounter{\yuyuspadecount}{\@yuyuspadecount}{9}

\newcommand*{\@yuyuclubcount}[1]{\ensuremath{
\ifcase #1\or\clubsuit_1\or\clubsuit_2\or\clubsuit_3
\or\clubsuit_4\or\clubsuit_5\or\clubsuit_6
\or\clubsuit_7\or\clubsuit_8\or\clubsuit_9
\else\@ctrerr\fi\relax}}
\newcommand*{\yuyuclubcount}[1]{%
\expandafter\@yuyuclubcount\csname c@#1\endcsname
}
\AddEnumerateCounter{\yuyuclubcount}{\@yuyuclubcount}{9}

\newcommand*{\@yuyustarcount}[1]{\ensuremath{
\ifcase #1\or\star_1\or\star_2\or\star_3
\or\star_4\or\star_5\or\star_6
\or\star_7\or\star_8\or\star_9
\else \@ctrerr \fi\relax}}
\newcommand*{\yuyustarcount}[1]{%
\expandafter\@yuyustarcount\csname c@#1\endcsname
}
\AddEnumerateCounter{\yuyustarcount}{\@yuyustarcount}{9}
\makeatother
%%%%%箇条書き環境



\usepackage{mandorasymb}
\usepackage{applekeys}
\renewcommand{\qedsymbol}{\pencilkey}
%\renewcommand{\qedsymbol}{\kinoposymbniko}




\usepackage{latexsym}
\DeclareMathOperator{\Hom}{Hom}
\DeclareMathOperator{\Isom}{Isom}
\DeclareMathOperator{\ISOM}{\mathbf{Isom}}
\DeclareMathOperator{\id}{\mathrm{id}}
\DeclareMathOperator{\im}{\mathrm{Im}}

\DeclareMathOperator{\coker}{\mathrm{coker}}
\DeclareMathOperator{\colim}{\mathrm{colim}}
\DeclareMathOperator{\plim}{\mathrm{lim}}
\DeclareMathOperator{\rank}{\mathrm{rank}}
\DeclareMathOperator{\codim}{\mathrm{codim}}

\DeclareMathOperator{\Spec}{\mathrm{Spec}}
\DeclareMathOperator{\Proj}{\mathrm{Proj}}
\DeclareMathOperator{\Sym}{\mathrm{Sym}}
\DeclareMathOperator{\Ext}{\mathrm{Ext}}
\DeclareMathOperator{\Bs}{\mathrm{Bs}}
\DeclareMathOperator{\Bl}{\mathrm{Bl}}
\DeclareMathOperator{\Sing}{\mathrm{Sing}}
\DeclareMathOperator{\red}{\mathrm{red}}
\DeclareMathOperator{\Reg}{\mathrm{Reg}}
\DeclareMathOperator{\Ridge}{\mathrm{Ridge}}
\DeclareMathOperator{\Hilb}{\mathrm{Hilb}}
\DeclareMathOperator{\Grass}{\mathrm{Grass}}


\newcommand{\A}{\mathbb{A}}
\newcommand{\C}{\mathbb{C}}
\renewcommand{\P}{\mathbb{P}}
\newcommand{\R}{\mathbb{R}}
\newcommand{\Q}{\mathbb{Q}}
\newcommand{\Z}{\mathbb{Z}}
\newcommand{\N}{\mathbb{N}}



\newcommand{\mcA}{\mathcal{A}}
\newcommand{\mcB}{\mathcal{B}}
\newcommand{\mcC}{\mathcal{C}}
\newcommand{\mcD}{\mathcal{D}}
\newcommand{\mcE}{\mathcal{E}}
\newcommand{\mcF}{\mathcal{F}}
\newcommand{\mcG}{\mathcal{G}}
\newcommand{\mcH}{\mathcal{H}}
\newcommand{\mcI}{\mathcal{I}}
\newcommand{\mcJ}{\mathcal{J}}
\newcommand{\mcK}{\mathcal{K}}
\newcommand{\mcL}{\mathcal{L}}
\newcommand{\mcM}{\mathcal{M}}
\newcommand{\mcN}{\mathcal{N}}
\newcommand{\mcO}{\mathcal{O}}
\newcommand{\mcP}{\mathcal{P}}
\newcommand{\mcQ}{\mathcal{Q}}
\newcommand{\mcR}{\mathcal{R}}
\newcommand{\mcS}{\mathcal{S}}
\newcommand{\mcT}{\mathcal{T}}
\newcommand{\mcU}{\mathcal{U}}
\newcommand{\mcV}{\mathcal{V}}
\newcommand{\mcW}{\mathcal{W}}
\newcommand{\mcX}{\mathcal{X}}
\newcommand{\mcY}{\mathcal{Y}}
\newcommand{\mcZ}{\mathcal{Z}}

\DeclareMathOperator{\OOO}{\mcO}

\newcommand{\OB}{{\OOO_B}}
\newcommand{\OC}{{\OOO_C}}
\newcommand{\OD}{{\OOO_D}}
\renewcommand{\OE}{{\OOO_E}}
\newcommand{\OF}{{\OOO_F}}
\newcommand{\OH}{{\OOO_H}}
\newcommand{\OP}{{\OOO_P}}
\newcommand{\OQ}{{\OOO_Q}}
\newcommand{\OR}{{\OOO_R}}
\newcommand{\OS}{{\OOO_S}}
\newcommand{\OT}{{\OOO_T}}
\newcommand{\OU}{{\OOO_U}}
\newcommand{\OV}{{\OOO_V}}
\newcommand{\OW}{{\OOO_W}}
\newcommand{\OX}{{\OOO_X}}
\newcommand{\OY}{{\OOO_Y}}
\newcommand{\OZ}{{\OOO_Z}}

\newcommand{\OO}[1]{\OOO_{#1}}

\newcommand{\loc}{\mathrm{loc}}

\newcommand{\rsa}{\rightsquigarrow}
\newcommand{\dto}{\dashrightarrow}
\renewcommand{\emptyset}{\varnothing}

\def\dfn{:\overset{\mbox{{\scriptsize def}}}{=}}



%%%%%%%%%タイトル
\title{Hartshorne Exercise II.8.2}
\author{ゆじ}


\begin{document}
\maketitle



このノートでは、\cite[演習 II.8.2]{Ha}に解答を与える。
\(k\)を基礎体とする。


\begin{Ex*}\label{ex}
  \(X\)を\(k\)上\(n\)次元の代数多様体、
  \(\mcE\)を\(X\)上のランク\(r > n\)の局所自由層、
  \(V\subset H^0(X,\mcE)\)を\(\mcE\)を生成する大域切断のなす部分空間とする。
  このとき、大域切断\(0\neq s\in V\)であって、
  対応する射\(s:\OX\to \mcE\)の余核が局所自由となるものが存在することを示せ。
\end{Ex*}

\(V\)を有限次元と仮定しても良いことに注意しておく:
\(f:V_X\to \mcE\)を\(V\subset H^0(X,\mcE)\)に対応する射とする。
仮定から\(f\)は全射である。
\(V = \bigcup_{i\in I}V_i\)と有限次元部分空間の和として表す。
\(f_i:V_{i,X}\to \mcE\)を
\(V_i\subset V\)の\(X\to \Spec(k)\)に沿った基底変換と\(f\)の合成とする。
このとき\(\bigcup_{i\in I}\im(f_i) = \mcE\)であるが、
\(X=\bigcup_{j\in J}U_j\)と有限個のアフィンスキームで覆えば、
各\(j\)に対して\(i_j\)が存在して\(\im(f_{i_j}|_{U_j})=\mcE|_{U_j}\)となることがわかり、
これらの\(i_j\)より大きな\(i\)をとることで、
十分大きな\(i\)に対して\(\im(f_i)=\mcE\)となることがわかる。
すなわち、ある有限次元部分空間\(V_0\subset V\)が存在して、
\(\mcE\)は\(V_0\)に属する大域切断で生成される。

\(V\)を有限次元として話を進める。

\begin{notation*}
  \
  \begin{itemize}
    \item
    基礎体を\(k\)と置く。
    \item
    スキームの射\(f:T\to S\)と\(S\)上の対象\(F\)
    (\(S\)-スキームや、\(S\)上のスキームの射や、\(S\)上の準連接層など)
    に対し、\(F_T\)や\(F|_T\)や\(f^*F\)で\(F\)の射\(T\to S\)による基底変換を表す。
  \end{itemize}
\end{notation*}



\section{大域切断の零点集合}
\label{section: S1}

\(n\)次元代数多様体\(X\)上の
ランク\(r\)の局所自由層\(\mcE\)と
\(\mcE\)を生成する大域切断のなす\(d\)-次元部分線形空間\(V\subset H^0(X,\mcE)\)を与える。
元\(s\in V\)を取り、
\(s\)の零点集合を調べる。

まず、\(s=0\)で定まる\(X\)の閉部分スキームがどのように構成されるか見る。
\[\OX \xrightarrow{s_X} V_X\to \mcE\]
を点\(p\in X\)に基底変換すると、
\[k(p) \xrightarrow{s} V \to \mcE_p\]
を得る。
\(s(p)=0\)の意味は、この射の列の合成が\(0\)射だということである。
双対をとることで、
\(s(p)=0\)は
\[
\mcE_p^{\vee}\to V^{\vee}\xrightarrow{s^{\vee}} k(p)
\]
の合成が\(0\)射であることと同値である。
従って、\(s=0\)という閉部分スキームは
\[
(s=0) \dfn \coker(\mcE^{\vee} \to V_X^{\vee} \xrightarrow{s^{\vee}_X} \OX)
\]
と定義される。

閉部分スキーム\((s=0)\)は、\(s\)を\(0\)でない定数倍で置き換えても変わらない注意する。
すると、各\(\P(V^{\vee})\)の元に対して\(X\)の閉部分スキームが定まることになり、
\(\P(V^{\vee})\)でパラメーター付けられた\(X\)の閉部分スキームの族
\[H\subset \P(V^{\vee})\times_kX\]
で、各\(s\in V\)に対して、対応する\(\P(V^{\vee})\)のfiberが
\(s=0\)となるもの、
の存在を期待したくなる。
この\(H\)を構成する。

\(\mcK\dfn \ker(V_X^{\vee}\to \mcE)\)と置くと、双対をとることで全射
\[
V_X^{\vee}\to \mcK
\]
を得る。
この全射は\(X\)上の射影束の閉埋め込み
\[
j:\P_X(\mcK^{\vee})\to \P(V^{\vee})\times_kX
\]
を定める。
\(j\)が求める\(H\subset \P(V^{\vee})\times_kX\)を定めることを示す。
元\(0\neq s\in V\)を取れば、一点からの射
\(\Spec(k)\xrightarrow{s} \P(V^{\vee})\)が定まり、
基底変換することで閉埋め込み\(i_s:X\to \P(V)^{\vee}\times_kX\)を得る。
\(j\)を\(i_s\)に沿って基底変換すると、
\(X\)の閉部分スキーム\(s_0:Y\subset X\)が定まる:
\[
\begin{CD}
  Y @>s_0>> X \\
  @VVV @Vi_sVV \\
\P_X(\mcK^{\vee}) @>j>> \P(V^{\vee})\times_kX.
\end{CD}
\]
この\(X\)上の図式を点\(p\in X\)まで基底変換すれば、
図式
\[
\begin{CD}
  Y_p @>>> p \\
  @VVV @VsVV \\
  \P(\mcK^{\vee}|_p) @>j_p>> \P(V^{\vee}).
\end{CD}
\]
を得る。
ここで\(Y_p\)は\(p\)または\(\emptyset\)である。
\[
Y_p\cong p \ \iff \ s\in \im(j_p) \ \iff s\in \im (\mcK|_p\to V) \
\iff \ s\in \ker(V\to \mcE_p)
\]
であるから、
\(Y\)は\(s=0\)で定まる閉部分スキームである。
\(Y\)は\(j\)と射影\(\P(V^{\vee})\times_kX\to \P(V^{\vee})\)の合成の
\(s\)が定める\(\P(V^{\vee})\)の点でのfiberであるから、
よって閉埋め込み\(j:\P_X(\mcK^{\vee})\to \P(V^{\vee})\times_kX\)は、
各\(s\in V\)に対して、対応する\(\P(V^{\vee})\)のfiberが
\(s=0\)となるような、
\(\P(V^{\vee})\)でパラメーターづけられた\(X\)の閉部分スキームの族を定める。







\section{問題の解答}


この節では、\autoref{section: S1}の議論を念頭において、問題に解答を与える。


\begin{proof}
  \(\mcE\)は自由層でないとして良い。

  \(d=\dim V\)と置くと、全射\(V_X\to \mcE\)の存在から\(d > r > n\)である。
  \(\mcK\dfn \ker (V_X\to \mcE)\)と置くと、\(\mcK\)のランクは\(d-r > 0\)である。
  双対をとることで全射\(V|_X^{\vee} \to \mcK^{\vee}\)を得る。
  この全射が引き起こす\(X\)上の射影束の閉埋め込み
  \[j:\P_X(\mcK^{\vee})\to \P(V^{\vee})\times_k X\]
  と射影\(\P(V^{\vee})\times_k X\to \P(V^{\vee})\)を合成すると、
  射
  \[
  f:\P_X(\mcK^{\vee})\to \P(V^{\vee})
  \]
  を得る。
  \(\dim (\P_X(\mcK^{\vee})) = d-r+n-1 < d-1 = \dim \P(V^{\vee})\)
  なので\(f\)は全射ではない。
  この\(f\)の像に入らない点\(\bar{s}\in \P(V^{\vee})\)
  を与える元\(0\neq s\in V\)を取れば、
  \(\bar{s}\)のfiberは\(\emptyset\)であるから、
  \((s=0)=\emptyset\)となる。
  ここで、元\(s\in V\)の定める射\(\OX\xrightarrow{s_X}V_X \to \mcE\)
  の双対
  \[\mcE^{\vee}\to V^{\vee}_X \xrightarrow{s_X^{\vee}} \OX\]
  の余核がちょうど\(s=0\)で定まる閉部分スキームの構造層であることに注意すると、
  今、\((s=0)=\emptyset\)であるから、
  \(\mcE^{\vee}\to V^{\vee}_X \xrightarrow{s_X^{\vee}} \OX\)
  の合成が全射であることがわかる。
  従って\(s\)が所望の大域切断である。
\end{proof}








\begin{thebibliography}{9}
  \bibitem[Ha]{Ha}
  R.Hartshorne,
  \textit{Algebraic Geometry}.
  Springer-Verlag, New Tork, 1977. Graduate Text in Mathematics, No. 52.
\end{thebibliography}



\end{document}
