\documentclass[uplatex]{jsarticle}

\usepackage{amssymb}
\usepackage{amsmath}
\usepackage{mathrsfs}
\usepackage{amsfonts}
\usepackage{mathtools}
\usepackage{stmaryrd}%%%%%べき級数のカッコ

\usepackage{xcolor}
\usepackage[dvipdfmx]{graphicx}



\newcommand{\myOriginalPackages}[1]{#1}

\myOriginalPackages{
\usepackage{applekeys}
\usepackage{mandorasymb}
}
\usepackage{ulem}
\usepackage{braket}
\usepackage{framed}


%%%%%ハイパーリンク
\usepackage[setpagesize=false,dvipdfmx]{hyperref}
\usepackage{aliascnt}
\hypersetup{
    colorlinks=true,
    citecolor=blue,
    linkcolor=blue,
    urlcolor=blue,
}
%%%%%ハイパーリンク




%%%%%図式
\usepackage{tikz}%%%図
\usetikzlibrary{arrows}
\usepackage{amscd}%%%簡単な図式
%%%%%図式


%%%%%%%%%%%%定理環境%%%%%%%%%%%%
%%%%%%%%%%%%定理環境%%%%%%%%%%%%
%%%%%%%%%%%%定理環境%%%%%%%%%%%%

\usepackage{amsthm}


\newcommand{\myTheoremEnvironments}[1]{#1}

\myTheoremEnvironments{
\theoremstyle{definition}
\newtheorem{thm}{定理}[section]
\newcommand{\thmautorefname}{定理}

\newaliascnt{prop}{thm}%%%カウンター「prop」の定義(thmと同じ)
\newtheorem{prop}[prop]{命題}
\aliascntresetthe{prop}
\newcommand{\propautorefname}{命題}%%%カウンター名propは「命題」で参照する

\newaliascnt{cor}{thm}
\newtheorem{cor}[cor]{系}
\aliascntresetthe{cor}
\newcommand{\corautorefname}{系}

\newaliascnt{lem}{thm}
\newtheorem{lem}[lem]{補題}
\aliascntresetthe{lem}
\newcommand{\lemautorefname}{補題}

\newaliascnt{defi}{thm}
\newtheorem{defi}[defi]{定義}
\aliascntresetthe{defi}
\newcommand{\defiautorefname}{定義}

\newaliascnt{prob}{thm}
\newtheorem{prob}[prob]{練習問題}
\aliascntresetthe{prob}
\newcommand{\probautorefname}{練習問題}


%%%%%%%番号づけない定理環境
\newtheorem*{exam*}{例}
\newtheorem*{rem*}{ノート}
\newtheorem*{question*}{疑問}
\newtheorem*{defi*}{定義}

\newtheorem*{nikki*}{日記}
\newtheorem*{thm*}{定理}

}

%%%証明環境を「proof」から「証明.」に変えるやつ
\renewcommand\proofname{\bf 解答.}
\renewcommand{\qedsymbol}{\kinoposymbniko}


%%%%%%%%%%%%定理環境%%%%%%%%%%%%
%%%%%%%%%%%%定理環境%%%%%%%%%%%%
%%%%%%%%%%%%定理環境%%%%%%%%%%%%




%%%%%箇条書き環境
\usepackage[]{enumitem}

\makeatletter
\AddEnumerateCounter{\fnsymbol}{\c@fnsymbol}{9}%%%%fnsymbolという文字をenumerate環境のパラメーターで使えるようにする。
\makeatother

\renewcommand{\theenumi}{(\arabic{enumi})}%%%%%itemは(1),(2),(3)で番号付ける。
\renewcommand{\labelenumi}{\theenumi}
\renewcommand{\theenumii}{(\alph{enumii})}
\renewcommand{\labelenumii}{\theenumii}
%%%%%箇条書き環境


\usepackage{latexsym}

\newcommand{\myMacros}[1]{#1}

\myMacros{
\DeclareMathOperator{\Hom}{Hom}
\DeclareMathOperator{\End}{End}
\DeclareMathOperator{\Isom}{Isom}
\DeclareMathOperator{\ISOM}{\mathbf{Isom}}
\DeclareMathOperator{\id}{\mathrm{id}}
\DeclareMathOperator{\im}{\mathrm{Im}}
\DeclareMathOperator{\Spec}{\mathrm{Spec}}
\newcommand{\Supp}{\mathrm{Supp}}
\DeclareMathOperator{\Aut}{\mathrm{Aut}}

\newcommand{\coker}{\mathrm{coker}}
\newcommand{\rad}{\mathrm{rad}}

\DeclareMathOperator{\Tor}{\mathrm{Tor}}
\DeclareMathOperator{\Ext}{\mathrm{Ext}}

\DeclareMathOperator{\colim}{\mathrm{colim}}
\DeclareMathOperator{\plim}{\mathrm{lim}}
\newcommand{\Ob}{\mathrm{Ob}}

\newcommand{\rsa}{\rightsquigarrow}
\renewcommand{\coprod}{\amalg}
\renewcommand{\emptyset}{\varnothing}
\newcommand{\ep}{\varepsilon}

\newcommand{\dfn}{:\overset{\mathrm{\scriptsize def}}{=}}
\newcommand{\deff}{:\hspace{-3pt}\overset{\text{def}}{\iff}}
\newcommand{\lb}[1]{\llbracket #1\rrbracket}
\newcommand{\dl}{\partial}
\DeclareMathOperator{\tr}{\mathrm{tr}}

\newcommand{\Sym}{\mathrm{Sym}}
\newcommand{\Mod}{\mathsf{Mod}}
\newcommand{\Ab}{\mathsf{Ab}}


\renewcommand{\ae}{\text{a.e.}}
\newcommand{\as}{P\text{-a.s.}}
\newcommand{\Var}{\mathrm{Var}}
\newcommand{\Cov}{\mathrm{Cov}}
\newcommand{\Dom}{\mathcal{D}om}
%\DeclareMathOperator{\liminf}{\textrm{lim inf}}



\newcommand{\A}{\mathbb{A}}
\newcommand{\C}{\mathbb{C}}
\newcommand{\D}{\mathbb{D}}
\newcommand{\E}{\mathbb{E}}
\newcommand{\F}{\mathbb{F}}
\newcommand{\N}{\mathbb{N}}
\renewcommand{\P}{\mathbb{P}}
\newcommand{\R}{\mathbb{R}}
\newcommand{\Q}{\mathbb{Q}}
\newcommand{\T}{\mathbb{T}}
\newcommand{\Z}{\mathbb{Z}}


\newcommand{\I}{\mathbf{1}}



\newcommand{\mcA}{\mathcal{A}}
\newcommand{\mcB}{\mathcal{B}}
\newcommand{\mcC}{\mathcal{C}}
\newcommand{\mcD}{\mathcal{D}}
\newcommand{\mcE}{\mathcal{E}}
\newcommand{\mcF}{\mathcal{F}}
\newcommand{\mcG}{\mathcal{G}}
\newcommand{\mcH}{\mathcal{H}}
\newcommand{\mcI}{\mathcal{I}}
\newcommand{\mcJ}{\mathcal{J}}
\newcommand{\mcK}{\mathcal{K}}
\newcommand{\mcL}{\mathcal{L}}
\newcommand{\mcM}{\mathcal{M}}
\newcommand{\mcN}{\mathcal{N}}
\newcommand{\mcO}{\mathcal{O}}
\newcommand{\mcP}{\mathcal{P}}
\newcommand{\mcQ}{\mathcal{Q}}
\newcommand{\mcR}{\mathcal{R}}
\newcommand{\mcS}{\mathcal{S}}
\newcommand{\mcT}{\mathcal{T}}
\newcommand{\mcU}{\mathcal{U}}
\newcommand{\mcV}{\mathcal{V}}
\newcommand{\mcW}{\mathcal{W}}
\newcommand{\mcX}{\mathcal{X}}
\newcommand{\mcY}{\mathcal{Y}}
\newcommand{\mcZ}{\mathcal{Z}}



\newcommand{\mfa}{\mathfrak{a}}
\newcommand{\mfb}{\mathfrak{b}}
\newcommand{\mfc}{\mathfrak{c}}
\newcommand{\mfd}{\mathfrak{d}}
\newcommand{\mfe}{\mathfrak{e}}
\newcommand{\mff}{\mathfrak{f}}
\newcommand{\mfg}{\mathfrak{g}}
\newcommand{\mfh}{\mathfrak{h}}
\newcommand{\mfi}{\mathfrak{i}}
\newcommand{\mfj}{\mathfrak{j}}
\newcommand{\mfk}{\mathfrak{k}}
\newcommand{\mfl}{\mathfrak{l}}
\newcommand{\mfm}{\mathfrak{m}}
\newcommand{\mfn}{\mathfrak{n}}
\newcommand{\mfo}{\mathfrak{o}}
\newcommand{\mfp}{\mathfrak{p}}
\newcommand{\mfq}{\mathfrak{q}}
\newcommand{\mfr}{\mathfrak{r}}
\newcommand{\mfs}{\mathfrak{s}}
\newcommand{\mft}{\mathfrak{t}}
\newcommand{\mfu}{\mathfrak{u}}
\newcommand{\mfv}{\mathfrak{v}}
\newcommand{\mfw}{\mathfrak{w}}
\newcommand{\mfx}{\mathfrak{x}}
\newcommand{\mfy}{\mathfrak{y}}
\newcommand{\mfz}{\mathfrak{z}}

\DeclareMathOperator{\OOO}{\mcO}

\newcommand{\OC}{\OOO_C}
\newcommand{\OD}{\OOO_D}
\renewcommand{\OE}{\OOO_E}
\newcommand{\OF}{\OOO_F}
\newcommand{\OH}{\OOO_H}
\newcommand{\OS}{\OOO_S}
\newcommand{\OT}{\OOO_T}
\newcommand{\OU}{\OOO_U}
\newcommand{\OV}{\OOO_V}
\newcommand{\OW}{\OOO_W}
\newcommand{\OX}{\OOO_X}
\newcommand{\OY}{\OOO_Y}
\newcommand{\OZ}{\OOO_Z}

\newcommand{\OO}[1]{\OOO_{#1}}
}



\allowdisplaybreaks[1]




\newcommand{\HereBeginTikz}{}
\newcommand{\HereEndTikz}{}
\newcommand{\HTMLhead}[1]{}

\HTMLhead{
---
layout: article-type
title: "p-adic Weierstrass Preparation Theorem"
category: Notes
tag: "Commtative Algebra"
author: Yujitomo
description: "p進Weierstrassの準備定理に関するノート"
---
}



\title{\(p\)-進Weiestrassの準備定理}
\date{\today}
\author{ゆじとも}

\begin{document}

\maketitle





このノートでは、p進Weierstrassの準備定理を証明します。
\href{https://adventar.org/calendars/6146}{Math Advent Calendar 2021}
用のノートです。
あまり難しい話題ばかりなのも良くないかなと思うので、やさしめな話題を選びました。
作業 (=苦労) をすれば証明できる定理ですが、
そのような苦労が可換環論の常識に帰着されることが嬉しい、という話です。


\begin{thm*}[p進Weierstrassの準備定理]
  \(A\)を完備離散付値環、\(\pi\)を素元、\(f\in A\lb{t}\)を\(0\)でないべき級数とする。
  このとき、あるdistinguishedな多項式\(p\)と単元\(u\in (A\lb{t})^{\times}\)とある\(n\geq 0\)が存在して、
  \(f = \pi^npu\)が成り立つ。
\end{thm*}

このノートでは上の定理を証明します。



\section{定義など}


\begin{defi}
  このノートでは、最高次係数が単元であるような多項式のことを\textbf{モニック}と言います。
  さらに、局所環係数のモニック\(f\)に対して、\(f\)の最高次以外の係数が極大イデアルの元であるとき、
  \(f\)は\textbf{distinguished}であると言います。
\end{defi}




\section{本題}


まず完備離散付値環に関する以下の定理を思い出します:

\begin{thm}[{cf. \cite[II. Proposition 3]{Serre}}]\label{loc flds thm}
  \(A\)を完備離散付値環、\(K\)を\(A\)の商体、\(L\)を\(K\)の有限次拡大とする。
  このとき、\(A\)の\(L\)での整閉包\(B\)は\(A\)上有限な完備離散付値環である。
\end{thm}

\begin{proof}
  簡単に説明を加えておく。
  \(A\)の完備性より、\(L\)の位相は\([L:K]\)個の\(K\)の直積位相と一致し、
  \(A\)の位相と両立的な\(L\)の付値が一意的であることがわかる。
  これから\(B\)が完備離散付値環であることが従う。
  \(B\)が\(A\)上有限であることが残っている。
  \(L/K\)が分離拡大であれば簡単 (cf. \cite[命題5.17]{AM}) なので、純非分離拡大の場合が問題となる。
  この場合は、\(\pi\in A\)を素元として、
  \(B/\pi B\)の\(A/(\pi)\)-線形空間としての基底の\(B\)上へ持ち上げが\(A\)上一次独立となる。
  それらで生成された部分\(A\)-加群\(E\subset B\)を考えると、
  任意の\(b\in B\)が\(b_0\in E\)と\(b_1\in B\)によって\(b=b_0+\pi b_1\)と表されるので、
  これを繰り返すと\(A\)の完備性によって\(E=B\)が従う。
\end{proof}


\begin{rem*}
  完備ネーター局所環はJapanese環であるという一般論もある。
  \cite[定理8.2.1]{Nag}参照。もしくは、
  \cite[\href{https://stacks.math.columbia.edu/tag/032W}{Tag 032W}]{stacks-project}と
  \cite[\href{https://stacks.math.columbia.edu/tag/0334}{Tag 0334}]{stacks-project}を組み合わせても良い。
\end{rem*}



\begin{cor}\label{monic 1}
  \(A\)を完備離散付値環、\(\pi\)を素元、\(K\)を\(A\)の商体、\(k=A/(\pi)\)を剰余体、\(L\)を\(K\)の有限次拡大、
  \(A\subset B\subset L\)を\(A\)上一元生成整域で、\(B\cong A[t]/(f)\)であるとする。
  このとき\(f\)はモニックである。
\end{cor}

\begin{proof}
  \(B\)は一次元なので、\(A\)の\(L\)での整閉包に含まれる。
  従って\(t\in B\)は\(A\)上整であり、その最小多項式\(f\)の最高次係数は単元である。
\end{proof}

系として以下が従います:

\begin{cor}\label{prime element is monic}
  \(A\)を完備離散付値環、\(f\in A[t]\)を素元とする。\(K\)を\(A\)の商体とする。
  \begin{enumerate}
    \item \label{prime 1}
    \(f\)はモニックである。
    \item \label{prime 2}
    \(f\in (\pi,t)\)であれば、\(f\)はdistinguishedである。
  \end{enumerate}
\end{cor}

\begin{proof}
  \ref{prime 1}は\autoref{monic 1}の直接の帰結である。
  \ref{prime 2}を示す。
  \(k \dfn A/(\pi)\)とおき、
  \(\bar{f}\)を\(k[t]\)での\(f\)の像とすると、
  \autoref{loc flds thm}より\(\Spec(k[t]/(\bar{f}))\)は一点集合である。
  \(f\in (\pi,t)\)であるから、\(\bar{f}\in (t)\)であり、
  よって\(f\)の最高次以外の係数は\(\pi\)で割り切れる。
\end{proof}


\begin{rem*}
  \(A[t]/(f)\)が\(A\)上完全分岐していれば、
  \(f\)はEisenstein多項式になります (cf. \cite[I. Proposition 18]{Serre})。
\end{rem*}




次に、\(A\)を完備ネーター局所環、\(\mfm\)をその極大イデアルとします。
\(t\)を不定元として、
べき級数環\(A\lb{t}\)について考えます。
\(\mfm\)と\(f\)で生成される多項式環\(A[t]\)のイデアルを\((\mfm,t)\)で表すと、
局所化\(A[t]_{(\mfm,t)}\)から自然な射\(A[t]_{(\mfm,t)}\to A\lb{t}\)が出ます。
この射は、ネーター局所環\(A[t]_{(\mfm,t)}\)をその極大イデアル\((\mfm,t)\)で完備化するという射ですから、
忠実平坦になっています (cf. \cite[命題10.14, 演習問題3.16, 演習問題3.18]{AM})。


\begin{lem}\label{lem: cplt f}
  \((A,\mfm)\)を完備ネーター局所環、\(f\in A[t]\)を多項式とする。
  \begin{enumerate}
    \item \label{lem: cplt f 1}
    \(A[t]/(f) \to A[t]_{(\mfm,t)}/(f)\)から完備化により引き起こされる射
    \[
    A\lb{t}/(f)\to \hat{A[t]_{(\mfm,t)}/(f)}
    \]
    は同型射である。
    \item \label{lem: cplt f 2}
    \(f\)がモニックであれば、
    \(A[t]/(f)\to A\lb{t}/(f)\)から局所化により引き起こされる射
    \[
    A[t]_{(\mfm,t)}/(f) \to A\lb{t}/(f)
    \]
    は同型射である。
  \end{enumerate}
\end{lem}

\begin{proof}
  \ref{lem: cplt f 1}は、
  \(A[t]_{(\mfm,t)}/(f)\)が有限生成\(A[t]_{(\mfm,t)}\)-加群であることから従う
  (cf. \cite[命題10.13]{AM})。
  \ref{lem: cplt f 2}の状況では、
  \(A[t]_{(\mfm,t)}/(f)\)は有限\(A\)-加群なので、\(A\)の完備性より、
  \(A[t]_{(\mfm,t)}/(f)\)は\((\mfm,t)\)-進完備であることが従う。
  よって自然な射\(A[t]_{(\mfm,t)}/(f) \to \hat{A[t]_{(\mfm,t)}/(f)}\)
  は同型射である。あとは\ref{lem: cplt f 1}より従う。
\end{proof}


\begin{question*}
  ネーターじゃない場合はどうなんでしょうね?
\end{question*}



\begin{cor}\label{cor div thm poly}
  \(A\)を完備離散付値環とする。
  このとき、\(A[t]\)の素元は\(A\lb{t}\)の素元でもあり、
  逆に\(A\lb{t}\)の素元はすべて\(A[t]\)の素元と単元の積で表すことができる。
\end{cor}

\begin{proof}
  \autoref{prime element is monic}と\autoref{lem: cplt f} \ref{lem: cplt f 2}より、
  \(\Spec(A\lb{t}) \to \Spec(A[t]_{(\pi,t)})\)は全単射となる。
  これは\autoref{cor div thm poly}を示している。
\end{proof}


\begin{thm}[p進Weierstrassの準備定理]
  \(A\)を完備離散付値環、\(\pi\)を素元、\(f\in A\lb{t}\)を\(0\)でないべき級数とする。
  このとき、あるdistinguishedな多項式\(p\)と単元\(u\in (A\lb{t})^{\times}\)とある\(n\geq 0\)が存在して、
  \(f = \pi^npu\)が成り立つ。
\end{thm}

\begin{proof}
  まず、\(A\lb{t}\)は二次元正則局所環\(A[t]_{(\pi,t)}\)の完備化なので正則局所環であり、とくにUFDである。
  よって、\(f\)は\(\pi^n\)と、いくつかの素元の積\(q = q_1,\cdots,q_r\)を用いて、
  \(f=\pi^nq_1\cdots q_r\)と表すことができる。
  さらに\autoref{prime element is monic}と\autoref{cor div thm poly}より、
  \(A\lb{t}\)の素元\(q_i\)は、
  distinguishedな既約多項式\(q_i'\in A[t]\)と
  単元\(u_i'\in (A\lb{t})^{\times}\)の積として\(q_i = q_i'u_i\)と表すことができる。
  \(q_1'\cdots q_r' = p\)とおくと、\(p\in A[t]\)はdistinguishedな多項式であり、
  \(u = u_1\cdots u_r\)とおくと、\(u\in (A\lb{t})^{\times}\)は単元である。
  代入すると\(f=\pi^npu\)となる。
  これは所望の結果である。
\end{proof}






\begin{thebibliography}{9}
  \bibitem[アティマク]{AM}
  \href{https://www.kyoritsu-pub.co.jp/bookdetail/9784320017917}{アティマク}.
  \bibitem[永田]{Nag}
  \href{https://www.kinokuniya.co.jp/f/dsg-08-EK-0257022}{永田可換環論}.
  \bibitem[Serre]{Serre}
  \href{https://link.springer.com/book/10.1007/978-1-4757-5673-9}{Local Fields}.
  \bibitem[Stacks]{stacks-project}
  \href{https://stacks.math.columbia.edu/}{\textit{Stacks Project}}.
\end{thebibliography}

\end{document}
