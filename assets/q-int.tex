\documentclass[uplatex]{jsarticle}


\usepackage{amssymb, amsmath, amsfonts, amsthm, amscd, mathrsfs, mathtools,
  stmaryrd, ulem, braket, latexsym, xparse
}
\usepackage[left=25mm,right=25mm]{geometry}

\usepackage{xcolor}
\definecolor{Mygray}{RGB}{230,230,230}
\definecolor{Mybblue}{RGB}{0,100,250}
\definecolor{Myred}{RGB}{255,80,100}
\definecolor{Myorange}{RGB}{200,100,0}

%%%%% hyperlink
\usepackage{hyperref,aliascnt}
\hypersetup{colorlinks=true, citecolor=Myorange, linkcolor=Myorange, urlcolor=Myorange}
\renewcommand{\eqref}[1]{\textcolor{blue}{(\ref{#1})}}

%%%%%% font
\usepackage[T1]{fontenc}  
\usepackage{roboto}

%%%%% diagram
\usepackage{tikz, tikz-cd}
\usetikzlibrary{calc, matrix, arrows, decorations.markings}

%%%%% enumerate
\usepackage{enumitem, moreenum}
\makeatletter
\newcommand*{\@TrumpCount}[1]{\ensuremath{
  \ifcase #1\or\spadesuit\or\clubsuit\else \@ctrerr \fi\relax}}
\newcommand*{\TrumpCount}[1]{%
  \expandafter\@TrumpCount\csname c@#1\endcsname}
\AddEnumerateCounter{\TrumpCount}{\@TrumpCount}{2}
\renewcommand{\theenumi}{(\roman{enumi})}
\renewcommand{\labelenumi}{\theenumi}
\renewcommand{\theenumii}{(\alph{enumii})}
\renewcommand{\labelenumii}{\theenumii}
\makeatother


%%%%%%%%%%%% theorem env %%%%%%%%%%%%
\makeatletter
\renewcommand{\sectionautorefname}{Section}
\renewcommand{\proofname}{\bf Proof.}
\newcommand{\Mytheorem}[2]{
  \newaliascnt{#1}{thm}\newtheorem{#1}[#1]{#2}\aliascntresetthe{#1}
  \expandafter\def\csname #1autorefname\endcsname{#2}
}
\NewDocumentCommand{\MyThm}{ >{\SplitList{,}} m }{\ProcessList{#1}{\MyThmEach}}
\def\MyThmEach#1{
  \def\@@MyThmEach{\relax #1\relax}
  \expandafter\ifx\@@MyThmEach\else
  \expandafter\@MyThmEach\@@MyThmEach\relax\fi
}
\def\@MyThmEach\relax#1=#2\relax{\Mytheorem{#1}{#2}}

\theoremstyle{definition}

\newtheorem{thm}{定理}
\newcommand{\thmautorefname}{定理}
\MyThm{prop=命題, cor=系, lem=補題, defi=定義, rem=注意, notation=記号}

%%%%%%% non-numbering
\newtheorem*{exam*}{Example}
\newtheorem*{comment*}{Comment}
\newtheorem*{rrem*}{Remark}
\newtheorem*{defi*}{Definition}

%%%%%%% alphabtical numbering
\newtheorem{thmA}{Theorem}[section]
\newcommand{\thmAautorefname}{Theorem}
\renewcommand{\thethmA}{\Alph{thmA}}

\newaliascnt{corA}{thmA}
\newtheorem{corA}[corA]{Corollary}
\aliascntresetthe{corA}
\newcommand{\corAautorefname}{Corollary}

\newtheorem{situation}{Situation}[section]
\newcommand{\situationautorefname}{Situation}
\renewcommand{\thesituation}{\TrumpCount{situation}}

\makeatother
%%%%%%%%%%%% theorem env %%%%%%%%%%%%




%%%%%%%%%%%% my definitions %%%%%%%%%%%%
\makeatletter

\renewcommand{\emptyset}{\varnothing}
\newcommand{\dfn}{:\overset{\mbox{{\rm\scriptsize def}}}{=}}
\newcommand{\deff}{:\hspace{-3pt}\overset{\text{def}}{\iff}}
\renewcommand{\amalg}{\sqcup}
\newcommand{\red}{\mathrm{red}}
\newcommand{\smsq}{\mathrel{\rule[1.5pt]{0.3em}{0.3em}}}
\newcommand{\rsa}{\rightsquigarrow}

% math operators
\NewDocumentCommand{\MyMathOpCommands}{ >{\SplitList{,}} m }{
  \ProcessList{#1}{\MyMathOpEach}
}
\def\MyMathOpEach#1{
  \def\@@MyMathOpEach{#1==\relax}
  \expandafter\ifx\@@MyMathOpEach\else
  \expandafter\@MyMathOpEach\@@MyMathOpEach\relax\fi
}
\def\@MyMathOpEach#1=#2=\relax{
  \ifx\relax#2\relax
  \expandafter\DeclareMathOperator\csname #1\endcsname{#1}\else
  \expandafter\DeclareMathOperator\csname #1\endcsname{\@@@MyMathOpEach #2}\fi
}
\def\@@@MyMathOpEach#1={#1}
\MyMathOpCommands{Hom, Isom, ISOM=\mathbf{Isom}, im=\mathrm{Im}, id, dom, codom, et=\mathrm{\acute{e}t}, pr,
coker, Spec, Aut, End, rk=\mathrm{Rank}, trdeg=\mathrm{tr.deg}}
\DeclareMathOperator*{\colim}{\mathrm{colim}}
\DeclareMathOperator*{\plim}{\mathrm{lim}}

% group and monoids 
\NewDocumentCommand{\MyMathRmCommands}{ >{\SplitList{,}} m }{
  \ProcessList{#1}{\MyMathRmEach}
}
\def\MyMathRmEach#1{
  \def\@@MyMathRmEach{#1==\relax}
  \expandafter\ifx\@@MyMathRmEach\else
  \expandafter\@MyMathRmEach\@@MyMathRmEach\relax\fi
}
\def\@MyMathRmEach#1=#2=\relax{
  \ifx\relax#2\relax
  \expandafter\newcommand\csname #1\endcsname{\mathrm{#1}}\else
  \expandafter\newcommand\csname #1\endcsname{\@@@MyMathRmEach #2}\fi
}
\def\@@@MyMathRmEach#1={#1}
\MyMathRmCommands{sh, gp, inte=\mathrm{int}, sat, tor, free, triv=\mathrm{triv}, op}

% universes
\newcommand{\univ}[1]{\mathbf{#1}}
\newcommand{\usm}{{\(\univ{U}\)-small}}
\newcommand{\vsm}{{\(\univ{V}\)-small}}
\newcommand{\wsm}{{\(\univ{W}\)-small}}

% Categories
\newcommand{\MyCatDef}[2][]{
  \ifx\relax#1\relax\expandafter\newcommand\csname #2\endcsname{\mathsf{#2}}
  \else\expandafter\newcommand\csname #2\endcsname{#1}\fi
  \expandafter\newcommand\csname #2U\endcsname{\csname#2\endcsname_{\univ{U}}}
  \expandafter\newcommand\csname #2V\endcsname{\csname#2\endcsname_{\univ{V}}}
  \expandafter\newcommand\csname #2W\endcsname{\csname#2\endcsname_{\univ{W}}}
  \expandafter\newcommand\csname #2Class\endcsname{\text{\robotolight #2}}
}
\NewDocumentCommand{\MyCategories}{ >{\SplitList{,}} m }{
  \ProcessList{#1}{\MyCatDef}}
\MyCategories{Alg,Ab,Ring,Grp,Mon,Top,Cat,Grpd,Site,Sch,Mor,Sh}
% \MyCatDef[\Sch^{\log}]{LSch}
\newcommand{\sfSet}{\mathsf{Set}}
\newcommand{\SetU}{\sfSet_{\univ{U}}}
\newcommand{\SetV}{\sfSet_{\univ{V}}}
\newcommand{\SetW}{\sfSet_{\univ{W}}}
\newcommand{\SetClass}{\text{\robotolight Set}}
\newcommand{\LSch}{\Sch^{\log}}
\newcommand{\LSchU}{\LSch_{\univ{U}}}
\newcommand{\LSchV}{\LSch_{\univ{V}}}
\newcommand{\LSchW}{\LSch_{\univ{W}}}
\newcommand{\LSchClass}{\textrm{\robotolight Sch}^{\log}}

% mathcal, mathbb, scr, bb, bf
\makeatletter
\newcommand{\MyMathcal}[1]{\@tfor\Ch@r:=#1\do{%
  \expandafter\edef\csname mc\Ch@r\endcsname{\noexpand\mathcal{\Ch@r}}%
  \expandafter\edef\csname scr\Ch@r\endcsname{\noexpand\mathscr{\Ch@r}}%
  \expandafter\edef\csname bf\Ch@r\endcsname{\noexpand\mathbf{\Ch@r}}%
  \expandafter\edef\csname mf\Ch@r\endcsname{\noexpand\mathfrak{m}_{\Ch@r}}%
  \expandafter\edef\csname bb\Ch@r\endcsname{\noexpand\mathbb{\Ch@r}}%
  \expandafter\edef\csname O\Ch@r\endcsname{\noexpand\mathcal{O}_{\Ch@r}}%
  \expandafter\edef\csname M\Ch@r\endcsname{\noexpand\mathcal{M}_{\Ch@r}}%
  \expandafter\edef\csname oM\Ch@r\endcsname{\noexpand\overline{\noexpand\mathcal{M}}_{\Ch@r}}
}}
\makeatother
\MyMathcal{ABCDEFGHIJKLMNOPQRSTUVWXYZ}
\newcommand{\mfm}{\mathfrak{m}}
\newcommand{\A}{\bbA}
\newcommand{\C}{\bbC}
\newcommand{\F}{\bbF}
\newcommand{\G}{\bbG}
\newcommand{\N}{\bbN}
\let\oldbackslashP\P
\renewcommand{\P}{\bbP}
\newcommand{\Q}{\bbQ}
\newcommand{\R}{\bbR}
\newcommand{\Z}{\bbZ}

\renewcommand{\OO}[1]{\mathcal{O}_{#1}}
\renewcommand{\MM}[1]{\mathcal{M}_{#1}}
\renewcommand{\oMM}[1]{\overline{\mathcal{M}}_{#1}}

\makeatother
%%%%%%%%%%%% my definitions %%%%%%%%%%%%


%%%%%%%%%%%%%%%%%%%%%%%%%%%%%%%%%%%%%%%%%%%%%%%%%%%%%%%%%%%%
%%%%%%%%%%%%%%%%%%%%%%%%%%%%%%%%%%%%%%%%%%%%%%%%%%%%%%%%%%%%
%%%%%%%%%%%%%%%%%%%%%%%%%%%%%%%%%%%%%%%%%%%%%%%%%%%%%%%%%%%%
%%%%%%%%%%%%%%%%%%%%%%%%%%%%%%%%%%%%%%%%%%%%%%%%%%%%%%%%%%%%
%%%%%%%%%%%%%%%%%%%%%%%%%%%%%%%%%%%%%%%%%%%%%%%%%%%%%%%%%%%%


\title{MonoidsのPush-outのQuasi-Integral性}
\author{ゆじ}

\begin{document}

\maketitle

モノイドは可換で単位元を持つとする。
演算は加法で表し、単位元は\(0\)と表記する。

\begin{defi}
  \(M\)をモノイドとする。
  \begin{itemize}
    \item 自明なモノイドを (軽微な記号の濫用により) \(0\)と表す。
    \item \(M^{\times} \dfn \{m\in M | \exists m'\in M, m'+m=0\}\). 
    \item \(M\)が\textbf{sharp}であるとは、\(M^{\times} = 0\)となることを言う。
    \item \(M^{\gp}\dfn (M\times M)/\sim\), ただし
    \[(m_1,m_2)\sim (m_1',m_2') \ \deff  \ \exists m\in M,  m_1+m_2'+m = m_1'+m_2+m.\]
    \item 
    \(\eta_M:M\to M^{\gp}\)を\(\eta_M(m) = \overline{(m,0)}\)で定める。
    これは\(M\)に関して函手的である。
    \item 
    モノイド\(M\)が\textbf{integral}であるとは、\(\eta_M\)が単射であることを言う。
    \item 
    \(M^{\inte}\dfn \im(\eta_M)\)と書く。
    これはどんなモノイド\(M\)に対してもintegralなモノイドとなり、さらに\(M\)に関して函手的である。
    \item 
    モノイド\(M\)が\textbf{pre-integral}であるとは、
    \(\eta_M|_{M^{\times}}: M^{\times} \to M^{\gp}\)が単射であることを言う。
    \item \(\Mon\)ですべてのモノイドのなす圏、\(\Ab\)ですべてのアーベル群のなす圏を表し、
    \(\amalg\)は\(\Mon\)の中でのpush-outを表す。
  \end{itemize}
\end{defi}


\begin{rem}\label{rem: Mon to Ab}
  \(\Ab\subset \Mon\)は\((-)^{\gp}\)を右随伴に持つ包含函手であるので、特に余極限と交換する。
\end{rem}



\begin{lem}\label{po in Mon surj}
  \(M\xleftarrow{f}N\xrightarrow{g}L\)をモノイドの図式とする。
  \(P\dfn M\amalg_NL\)と置く (ただしこれは\(\Mon\)におけるpush-outである)。
  このとき、自然な射\(q: M\amalg L \to M\amalg_NL\)は全射である。
\end{lem}

\begin{proof}
  \(Q \dfn \im(q)\)と置き、\(i:Q\hookrightarrow P\)を包含射、
  \(q':M\amalg L \to Q\)を\(q = i\circ q'\)となる射とする。
  もし\(q\)が全射ではないとすると、push-outの普遍性より、
  ある\(r:P\to Q\)が存在して\(q' = r\circ q\)が成り立つ。
  \(q'\)は全射であるから、\(r\)も全射である。
  また、\(i\circ r \circ q = i\circ q' = q\)であるから、
  push-outの普遍性より\(i\circ r = \id_P\)が成り立つ。
  よって\(r\)は単射でもある。
  従って\(r\)は同型射となり、\(P=Q\)が従う。
  以上で\autoref{po in Mon surj}の証明を完了する。
\end{proof}




\begin{lem}\label{lem: str of gp}
  \(M\)をモノイドとする。
  \begin{enumerate}
    \item \label{enumi: mm is 0}
    任意の\(m\in M\)に対し、\((m,m)\in M\times M\)の定める同値類\([(m,m)]\in M^{\gp}\)は\(0\)である。
    \item \label{enumi: elements of gp}
    任意の\(m\in M^{\gp}\)に対し、
    ある\(a,b\in M\)が存在し、\(m + \eta_M(a) = \eta_M(b)\)となる。
    \item \label{enumi: if eq in gp}
    任意の\(a,b\in M\)に対し、
    \(\eta_M(a) = \eta_M(b)\)であるならば、
    ある\(c\in M\)が存在して\(a+c=b+c\)が成り立つ。
  \end{enumerate}
\end{lem}

\begin{proof}
  \ref{enumi: mm is 0}は、
  \(m+0 = 0+m\)であることと同値関係の定義より従う。
  \ref{enumi: elements of gp}は、
  \(m=[(b,a)]\)と表すことによって\(m = \eta_M(b) - \eta_M(a)\)となるので、これから帰結する
  (\ref{enumi: mm is 0}より\(-\eta_M(a) = [(0,a)]\)となることに注意)。
  \ref{enumi: if eq in gp}は、
  同値関係の定義より、\([(a,0)] = [(b,0)]\)であるとすると、ある\(c\in M\)が存在して
  \(a+c=a+0+c=0+b+c=b+c\)となるので、このことから帰結する。
  以上で\autoref{lem: str of gp}の証明を完了する。
\end{proof}




\begin{defi}[Quasi-integral]
  モノイド\(M\)が\textbf{quasi-integral}であるとは、
  任意の\(a,b\in M\)に対して、以下が成り立つことを言う: 
  \[a+b=a \ \ \Rightarrow \ \ b = 0.\]
\end{defi}


\begin{lem}\label{lem: basic props *-ints}
  \(M\)をモノイドとする。
  \begin{enumerate}
    \item \label{enumi: q-int iff triv ker}
    \(M\)がquasi-integralであるための必要十分条件は、
    \(\eta_M^{-1}(0) = 0\)となることである。
    \item \label{enumi: int implies q-int implies pre-int}
    integral \ \(\Rightarrow\) \ quasi-integral \ \(\Rightarrow\) \ pre-integral. 
    \item \label{enumi: q-int intgralification is sharp}
    \(M\)がquasi-integralであれば、\(M\to M^{\inte}\)による\((M^{\inte})^{\times}\)の逆像はちょうど\(M^{\times}\)である。
  \end{enumerate}
\end{lem}

\begin{proof}
  \ref{enumi: q-int iff triv ker}を示す。
  まず必要性を示す。
  \(M\)がquasi-integralであるとして、\(a\in M\)が\(\eta_M(a) = 0\)を満たすとする。
  このとき、\(\eta_M(a) = 0 = \eta_M(0)\)であるから、
  \autoref{lem: str of gp} \ref{enumi: if eq in gp}より
  ある\(c\in M\)が存在して\(a+c = c\)となる。
  \(M\)はquasi-integralなので、\(a=0\)が従う。以上で必要性の証明を完了する。
  次に十分性を証明する。
  \(\eta_M^{-1}(0) = 0\)であると仮定して、
  \(a,b\in M\)が\(a+b=a\)を満たすとする。
  このとき、\(\eta_M(a) + \eta_M(b) = \eta_M(a)\)が成り立つので、
  \(\eta_M(b) = 0\)である。
  従って\(b\in \eta_M^{-1}(0) = 0\)となり\(b=0\)である。
  以上で\ref{enumi: q-int iff triv ker}の証明を完了する。

  \ref{enumi: int implies q-int implies pre-int}を示す。
  \(\eta_M\)が単射であれば\(\eta_M^{-1}(0)=0\)であるから、
  \ref{enumi: q-int iff triv ker}より「integral\(\Rightarrow\)quasi-integral」が従う。
  \(\ker(\eta_M|_{M^{\times}}) \subset \eta_M^{-1}(0)\)であるから、
  \ref{enumi: q-int iff triv ker}より「quasi-integral\(\Rightarrow\)pre-integral」が従う。
  以上で\ref{enumi: int implies q-int implies pre-int}の証明を完了する。

  \ref{enumi: q-int intgralification is sharp}を示す。
  \(m\in M\)が\(\eta_M(m)\in (M^{\inte})^{\times}\)を満たしているとする。
  すると\(\exists(-\eta_M(m))\in M^{\inte} = \im(\eta_M)\)であるから、ある\(m'\in M\)が存在して\(\eta_M(m') = -\eta_M(m)\)が成り立つ。
  両辺に\(\eta_M(m)\)を足すと\(\eta_M(m+m') = 0\)が成り立ち、\(M\)のquasi-integral性と\ref{enumi: q-int iff triv ker}より\(m+m'=0\)が成り立つ。
  これは\(m'\in M^{\times}\)を示している。
  \(M^{\times}\)の元が\((M^{\inte})^{\time}\)へ写されることは明らかである。
  以上で\autoref{lem: basic props *-ints}の証明を完了する。
\end{proof}


\begin{lem}\label{lem: int to q-int, sharpness pb}
  \(f:M\to P\)をモノイドの射とする。
  このとき、以下が成り立つ: 
  \begin{enumerate}
    \item \label{enumi: preint inj}
    \(P\)がpre-integralであれば、自然な射\(P^{\times}\times_P M \to P^{\times}\times_{P^{\gp}}M\to M\)は単射である。
    \item \label{enumi: preint and qint sharp}
    \(P\)がpre-integralであり、さらに\(M\)がquasi-integralであれば、自然な射\(P^{\times}\times_{P^{\gp}}M\to M\)の像は\(M^{\times}\)の部分群である。
    とくに、さらに\(M\)がsharpであれば、\(P^{\times}_PM\)は
    \item \label{enumi: qint isom}
    \(P\)がquasi-integralであれば、
    自然な射\(P^{\times}\times_P M \to P^{\times}\times_{P^{\gp}}M\)は同型射である。
  \end{enumerate}
\end{lem}

\begin{proof}
  \ref{enumi: preint inj}は\(P^{\times}\to P^{\gp}\)が単射であることから従う。
  \ref{enumi: preint and qint sharp}を示す。
  
  まず、\autoref{lem: basic props *-ints}\ref{enumi: int implies q-int implies pre-int}より\(P\)はpre-integralである。
  従って\(P^{\times}\to P^{\gp}\)は単射であり、\(P^{\times}\times_P M \to P^{\times}\times_{P^{\gp}}M\)は\(M\)の部分モノイドの間の包含射とみなすことができる (単射である)。
  よって\autoref{lem: int to q-int, sharpness pb}を示すためには、この射が全射であることを証明することが十分である。
  \(m\in M\)が\(f^{\gp}(m)\in \eta_P(P^{\times})\)を満たすとする。
  示すべきことは、\(f(m)\in P^{\times}\)となることである。
  \(f^{\gp}(m)\in \eta_P(P^{\times})\)であるから、ある\(p\in P^{\times}\)が存在して
  \(\eta_P(p) = f^{\gp}(m)\)となる。
  \(\eta\)の函手性より、\(f^{\gp}(m) = \eta_P(f(m))\)となる。
  ここで\(p\in P^{\times}\)であるから、\(\exists [-p]\in P\)である。
  この\(-p\in P\)を\(\eta_P(p) = f^{\gp}(m) = \eta_P(f(m))\)の両辺に足すことで、\(0=\eta_P(p) + \eta_P(-p) = \eta_P(f(m)) + \eta_P(-p) = \eta_P(f(m)-p)\)が成り立つ。
  \(P\)はquasi-integralであるから、
  \autoref{lem: basic props *-ints} \ref{enumi: q-int iff triv ker}より、
  \(f(m)-p\in \eta_P^{-1}(0)=0\)となる。
  よって\(f(m)=p\)であり、\(f(m)\in P^{\times}\)が従う。
  
  また、\(M\)がsharpでintegralであるときは、
  \(P^{\times}\times_P M \to P^{\times}\times_{P^{\gp}}M \to M\)を\(M^{\gp}\)の部分モノイドの間の包含射とみなすことで
  \[
    P^{\times}\times_P M \subset P^{\times}\times_{P^{\gp}}M = (P^{\times}\times_{P^{\gp}}M^{\gp})\times_{M^{\gp}}M = (P^{\times}\times_{P^{\gp}}M^{\gp})\cap M \subset M^{\times}
  \]
  が成り立つので、最後の主張はこれより従う。
  以上で\autoref{lem: int to q-int, sharpness pb}の証明を完了する。
\end{proof}



\begin{prop}[中山の呪い, {cf. \cite[Lemma 2.2.6]{Nak}}]\label{prop: curse of Nak}
  \(M\xleftarrow{f}N\xrightarrow{g}L\)をモノイドの図式とする。
  \(P\dfn M\amalg_NL\)と置く (ただしこれは\(\Mon\)におけるpush-outである)。
  次の主張について考える: 
  \begin{enumerate}[label=(\alph)]
    \item \label{P is q-int}
    \(P\)はquasi-integralである。
    \item \label{P is q-int}
    \(P\)はquasi-integralである。
    \item \label{assert only M,N,L}
    任意の\(n\in N^{\gp}\)に対して、\(f^{\gp}(n)\in \im(\eta_M)\)と\(g^{\gp}(-n)\in \im(\eta_L)\)が成り立てば、
    \(f^{\gp}(n) = 0\)と\(g^{\gp}(-n) = 0\)が帰結する。
  \end{enumerate}
  このとき、「\ref{assert only M,N,L}\(\Rightarrow\)\ref{P is q-int}」が成り立つ。
  また、\(M\)と\(L\)がsharpかつintegralであれば、
\end{prop}

\begin{proof}
  まず「\ref{assert only M,N,L} \(\Rightarrow\) \ref{P is q-int}」を証明する。
  \(p\in P\)が\(\eta_P(p) = 0\)を満たしているとする。
  \(p=0\)を示せば良い。
  \(i_M:M\to P\)と\(i_L:L\to P\)を自然な射とする。
  \autoref{po in Mon surj}より、
  ある\(m\in M, l\in L\)が存在して\(p = i_M(m) + i_L(l)\)が成り立つ。
  \(\eta\)の函手性より\(i_M^{\gp}(\eta_M(m)) + i_L^{\gp}(\eta_L(l)) = \eta_P(p) = 0\)が成り立つから、
  \autoref{rem: Mon to Ab}より、
  ある\(n\in N^{\gp}\)が存在して、\(\eta_M(m) = f^{\gp}(n)\)と\(\eta_L(l) = g^{\gp}(-n)\)が成り立つ。
  ここで\ref{assert only M,N,L}より、\(m=0\)と\(l=0\)が従う。
  これは\(p = i_M(m) + i_L(l) = 0\)を導く。
  以上で「\ref{assert only M,N,L} \(\Rightarrow\) \ref{P is q-int}」の証明を完了する。
  
  次に「\ref{P is q-int} \(\Rightarrow\) \ref{assert only M,N,L}」を証明する。
  \(P\)がquasi-integralであると仮定し、
  \(n\in N^{\gp}\)が\(f^{\gp}(n)\in \im(\eta_M)\)と\(g^{\gp}(-n)\in L^{\gp}\)を満たすとする。
  \(\eta_M(m) = f^{\gp}(n)\)となる\(m\in M\)と\(\eta_L(l) = g^{\gp}(-n)\)となる\(l\in L\)をとる。
  \(h = i_M \circ f = i_N\circ g\)と置く。
  このとき、\(\eta\)の函手性より、
  \[\eta_P(i_M(m) + i_L(l)) = i_M^{\gp}(f^{\gp}(n)) + i_L^{\gp}(g^{\gp}(-n)) = h^{\gp}(n + (-n)) = 0\]
  となる。
  よって\ref{P is q-int} (\(P\)のquasi-integral性) と
  \autoref{lem: basic props *-ints} \ref{enumi: q-int iff triv ker}より、\(i_M(m) + i_L(l) = 0\)が成り立つ。
  とくに、\(i_M(m),i_L(l)\in P^{\times}\)が成り立つ。
  ここで\(M,L\)はどちらもsharpかつintegralであるから、
  \autoref{lem: int to q-int, sharpness pb}より、
  \(m\in M\times_P P^{\gp} = 0\)と
  \(l\in L\times_P P^{\gp} = 0\)が従い、
  よって\(f^{\gp}(n) = \eta_M(m) = 0\)と\(g^{\gp}(-n) = \eta_L(l) = 0\)が帰結する。
  以上で\autoref{prop: curse of Nak}の証明を完了する。
\end{proof}






\begin{thebibliography}{9}
  \bibitem[Nak]{Nak}
  C. Nakayama,
  \textit{Log \'{E}tale Cohomology},
  Math. Ann. \textbf{308} (1997), 365-404.
\end{thebibliography}



\end{document}